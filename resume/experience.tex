\cvsection{Experiencia laboral}

\begin{cventries}
  \cventry
    {Analista de Software}
    {Plexus Tech}
    {Santiago de Compostela, España}
    {Ene. 2019 - Presente}
    {
      \begin{cvitems}
        \item {Analista de Software y Tech Lead en un equipo de 6 personas en el Área de Financiero de Inditex.}
        \item {Gestión y mantenimiento de dos aplicaciones legacy enmarcadas dentro del área financiera de la compañía. Java, Spring Boot, JMS, SWT, JavaFX, DB2, AS400, SQL Server, Jira, Bamboo, etc.}
        \item {Análisis técnico y funcional para el desarrollo de una solución que permita la integración de los sistemas legacy de Inditex con una herramienta comercial dedicada a la conciliación bancaria. En el desarrollo del proyecto he tenido que ejercer como arquitecto de software para proponer soluciones técnicas a los problemas y necesidades planteadas por el cliente.}
        \item {Soporte a otros proyectos internos (Plexus) en tecnologías como Android, Java, Python, Angular, Ionic, Kafka, Spark, Hadoop, Cassandra, Docker, etc.}
      \end{cvitems}
    }

  \cventry
    {Ingeniero de Software}
    {}
    {}
    {Sep. 2018 - Ene. 2019}
    {
      \begin{cvitems}
        \item {Diseño e implementación de una herramienta de análisis y monitorización de encuestas para hoteles, desde la arquitectura, el modelo de datos, la API REST y la aplicación web. MariaDB, Java, Spring Boot, JPA, API REST, Angular, D3.js, Bootstrap 4, Gulp.js, Nginx, Apache Tomcat, Docker, Docker Compose, etc.}
        \item {Migración de una plataforma de análisis de datos basada en información obtenida a partir de APs a tecnologías y entornos de Big Data. El objetivo fue el diseño de la arquitectura necesaria para la migración y optimización de los diferentes procesos llevados a cabo y la implementación de los mismos. Creación de servicios de ingesta, datalake, procesamiento, almacenamiento, visualización de los datos y despliegue. Cloudera, Hadoop, Spark, Kafka, Ignite, Redis, Cassandra, HBase, Docker, Java, Scala, Python, Aruba ALE, Angular, Spring Boot, Node.js, Microsoft Azure, etc.}
        \item {Soporte a otros proyectos internos en tecnologías como Android, Java, Python, Angular, Ionic, etc.}
      \end{cvitems}
    }

  \cventry
    {Desarrollador Full Stack}
    {}
    {}
    {Nov. 2016 – Sep. 2018}
    {
      \begin{cvitems}
        \item {Diseño, refactorización y ampliación de una plataforma de análisis de datos basada en información obtenida a partir de APs. He trabajado en todas las capas de la misma, desde la gestión de la base de datos, la implementación y refactorización del backend hasta la materialización del frontal web y su despliegue.}
        \item {Bases de datos columnares (MariaDB Columnstore), Java, JavaScript, jQuery, D3.js, HTML5, SASS, Bootstrap 4, Angular, Aruba ALE, Aruba ClearPass, Aruba Airwave, Docker, procesamiento de grandes volúmenes de datos, etc.}
      \end{cvitems}
    }

  \cventry
    {Desarrollador de Software Independiente}
    {Independiente}
    {Vigo, España}
    {Ene. 2016 - Sep. 2016}
    {
      \begin{cvitems}
        \item {Ampliación funcional de la herramienta de análisis de red desarrollada durante la estancia en Gradiant. Uso de tecnologías relacionadas con el almacenamiento, procesado y visualización de grandes cantidades de datos. Elasticsearch, Kibana, Grafana, MongoDB, Python, Qt, Linux y diversas tecnologías de red, Java, Spring, Javascript, Angular, Node.js, MariaDB, MongoDB, SonarQube, GitLab, etc.}
      \end{cvitems}
    }

  \cventry
    {Becario I+D+i}
    {Gradiant}
    {Vigo, España}
    {Sep. 2015 - Ene. 2016}
    {
      \begin{cvitems}
        \item {Desarrollo e implementación de una aplicación de análisis de red y estimación de QoS para plataformas GNU/Linux. Python, Qt, Linux, herramientas de red, etc.}
        \item {Implementación de un configurador y un compositor gráfico web para la gestión y visualización de grandes cantidades de datos. Múltiples librerías de creación de gráficos avanzados en Javasscript, HTML5, CSS3, jQuery, MongoDB, NodeJS, Jade, etc.}
        \item {Soporte y ayuda en otros proyectos. APIs REST, implementación de una aplicación para dispositivos Android con Android Studio, HTML5, CSS3, jQuery, Linux, redes, etc.}
      \end{cvitems}
    }
\end{cventries}
