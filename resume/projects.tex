\cvsection{Proyectos}

\begin{cventries}
  \cventry
    {Creador}
    {\href{https://skapesquad.com}{Skape Squad}}
    {Galicia, España}
    {Abr. 2018 - Presente}
    {
      \begin{cvitems}
        \item {Skape Squad da nombre a un grupo de cuatro amigxs que deciden embarcarse en una nueva aventura relacionada con los escape room, simplemente por hobby y diversión. Como dice nuestro perfil en Instagram (@skapesquad), "somos millennials aficionados a los Escape Room".}
        \item {Skape Squad se materializa en una página web/blog que pretende proporcionar información y opiniones sobre las salas de escape que vamos jugando por toda la geografía española. Pero no se trata tan solo de una web, sino que representa un sentimiento común de inquietud, curiosidad y ganas de viajar.}
        \item {Tecnológicamente, la aplicación de Skape Squad se desarrolló utilizando un stack con las siguientes tecnologías/frameworks:}
          \begin{itemize}[label={---}]
            \item {La información se persiste en una base de datos no relacional, MongoDB.}
            \item {Se utiliza una API REST como interfaz de obtención de datos, creada con Node.js + Express.}
            \item {Angular es el framework elegido para la creación de la aplicación web (PWA, SSR, SEO, lazy loading de imágenes, Google Maps, optimizada siguiendo las recomendaciones de Google Lighthouse, etc.).}
            \item {La aplicación está contenerizada y desplegada utilizando Docker.}
          \end{itemize}
      \end{cvitems}
    }

    \cventry
    {Creador de la aplicación web}
    {\href{https://epigraphica30.com}{Epigraphica 3.0}}
    {Galicia, España}
    {Ene. 2018 - Presente}
    {
      \begin{cvitems}
        \item {“Epigraphica 3.0 : Hacia la creación y diseño de un corpus digital de inscripciones romanas de la provincia de Ourense” (ED481D2017/013)}
        \item {Proyecto de investigación financiado por la Consellería de Cultura, Educación e Ordenación Universitaria de la Xunta de Galicia, con la colaboración de la Universidade de Santiago de Compostela, el Consello da Cultura Galega y el Archivo Epigráfico de Hispania. Dirigido por David Espinosa Espinosa (Doctor en Historia Antigua) y desarrollado por un equipo de investigación interdisciplinar integrado, además de por él, por Miguel Carrero Pazos (Doctor en Prehistoria) y Borja Paz Rodríguez (Ingeniero de Tecnologías de Telecomunicación).}
        \item {Epigraphica 3.0 tiene como finalidad la creación de un corpus epigráfico multilingüe, online y en open access a partir de la elaboración de una base de datos relacional y la implementación de un visor cartográfico y un visualizador de modelos 3D. El área objeto de estudio es la provincia de Ourense, territorio que, junto con Pontevedra, documenta el mayor número de inscripciones romanas de Galicia, lo que la convierte en una región de enorme interés para el estudio de los aspectos sociales, políticos, económicos y culturales de la presencia romana en el noroeste de Hispania. Por limitaciones temporales, los miliarios han sido excluidos de este corpus. Por causas ajenas a Epigraphica 3.0, no está disponible la información relativa a un conjunto amplio de inscripciones.}
        \item {Tecnológicamente, la aplicación de Epigraphica 3.0 se desarrolló utilizando un stack con las siguientes tecnologías/frameworks:}
          \begin{itemize}[label={---}]
            \item {La información se persiste en una base de datos relacional, MariaDB.}
            \item {Se utiliza una API REST como interfaz de obtención de datos, creada con Node.js + Express.}
            \item {Angular es el framework elegido para la creación de la aplicación web (PWA, SSR, SEO, lazy loading de imágenes, Google Maps, optimizada siguiendo Google Lighthouse, etc.).}
            \item {La aplicación está contenerizada y desplegada utilizando Docker.}
          \end{itemize}
      \end{cvitems}
    }

    \cventry
    {Creador}
    {Plataforma de análisis de red y estimación de QoS}
    {Galicia, España}
    {Sep. 2015 – Sep. 2016}
    {
      \begin{cvitems}
        \item {Plataforma (aplicación de escritorio) para entornos GNU/Linux que, mediante la realización de un conjunto de medidas configurables (ancho de banda, latencia, pérdidas, jitter, etc.), permite estimar la Calidad de Servicio (QoS) ofrecida por una determinada red. Los resultados son almacenandos y representandos de forma gráfica a través de un conjunto de visualizaciones configurables en Grafana.}
        \item {La aplicación está escrita en Python + Qt utilizando el binding PyQt. Los resultados se almacenan en Elasticsearch y se representan en Grafana.}
      \end{cvitems}
    }
\end{cventries}
