\cvsection{Resumen}

\begin{cvparagraph}
¡Hola! Soy Borja, un simple teleco apasionado de la programación, de los cachivaches, de la tecnología, de las redes, de los escape room... ¡Sí, sí! De los escape room, has oído bien. Gracias a este hobby, que me regaló tantos dolores de cabeza como momentos únicos, pude fortalecer todos los conocimientos que adquirí durante mi etapa académica y descubrir muchas más cosas que hoy en día dan vueltas por mi cabeza. ¡Todo empezó con una maldita página web! A partir de ese momento el interés creció aún más dentro de mí. ¿Cómo hacer algo de la mejor forma posible? ¿Cuál es la tecnología más adecuada? ¿Estaré siguiendo las mejores prácticas? ¿Estaría el tito Bob (Uncle Bob) orgulloso de mí?

Mis inquietudes por saber se centran en las buenas prácticas, los patrones de diseño, la aplicación de nuevas arquitecturas, la automatización de procesos, la CI/CD, la mejora de rendimiento, la optimización de procesos, el aprendizaje de nuevos lenguajes de programación y habilidades, la monitorización, la resolución de problemas, el uso de DDD, etc. No sería capaz de elegir una tecnología en concreto. Durante mi etapa profesional he ido cambiando y combinando muchas ramas, lo cual es gratificante y sirve como ejercicio de adaptación al cambio. A pesar de ello, guardo un especial cariño a tecnologías como Java, Spring, JavaScript, Angular, Node.js, Python, Docker, etc. También me suelo entretener bastante realizando labores propias de un DevOps, buscando siempre la forma más sencilla y práctica de realizar las cosas.

Me apasiona poder dedicar tiempo a pequeños proyectos personales que puedan ser de utilidad a otras personas que tengan las mismas inquietudes que yo. Por ejemplo, unos prácticos dotfiles para hacer más sencilla la configuración y personalización de sistemas operativos basados en GNU/Linux o una mini herramienta para gestionar y configurar servidores domésticos utilizando Docker, Docker Compose, Make y FZF.

En 2016 vi en redes sociales que el Doctor René Heller, astrofísico en el Max Planck Institute for Solar System Research, proponía un reto que consistía en descifrar un presunto mensaje alienígena. Con mucha curiosidad y trabajo conseguí averiguar lo que escondía y me convertí en uno de los 66 crackers exitosos del SETI Decrypt Challenge.
\end{cvparagraph}
